\documentclass[10pt,a5paper,fullpage]{book}
\usepackage[top=3.5cm,bottom=2.7cm,left=3cm,right=2.5cm]{geometry}
\usepackage[utf8]{inputenc}
\usepackage{amsmath}
\usepackage{textcomp}
\usepackage{verbatim}			
\usepackage{hyperref} 			% To generate hyperlinks for footnotes (disable before printing to old printer)
\usepackage{amsfonts}
\usepackage{amssymb}
\usepackage{graphicx}
\usepackage{chemfig}			% For chemists: to represent molecules
\usepackage[normalem]{ulem}
\usepackage[T1]{fontenc} 		% If you want to use guillemotleft/right
%\newcommand\fnurl[2]{ 			% (?)To use Web URL in footnotes
%	\href{#2}{#1}\footnote{\url{#2}} 
%}
% Below: style for hyperlinks and style
\hypersetup{
	pdftitle={Recueil de blagues},
	pdfauthor={Bob Sleigh (WS)},
	colorlinks=true,
	linkcolor=blue,
	filecolor=magenta,      
	urlcolor=cyan,
}

\author{Fait avec amour par Bob Sleigh}
\title{Recueil de blagues}
\date{2020-09-12}
\begin{document}
	\maketitle
	\tableofcontents
	\newpage
	\section*{Préface}
	\guillemotleft~\textit{C'est une blague, j'espère}~?~\guillemotright\\~\rightline{--- La réaction d'un ami en voyant ce recueil.} \\ \\
	\guillemotleft~\textit{Rire est le propre de l'Homme.}~\guillemotright\\~\rightline{--- Aristote, avant d'avoir lu cet ouvrage.} \\
	
	\underline{Mise en situation}~: vous êtes coincés dans un train/avion ou repas en famille et votre téléphone \sout{a déclaré forfait} vient de rendre son dernier soupir. Vous ne savez pas quoi faire et il reste encore quatre heures à tuer. Vous déprimez. \\ \\
	Et là, la solution miracle arrive~! \\ \\
	Ce recueil de blagues vous donnera le sourire dès le premier calembour (ou vous rassurera sur votre niveau de blagues) et vous permettra de faire rire vos voisins d'une manière ultra simple~! \\Ce livret\footnote{Écrit en \LaTeX~!} peut délivrer 420 blagues (\textit{nice}). % Plus des blagues qui n'ont pas été publiées : si vous lisez cela, vous pourrez y accéder ! 
	Si certaines sont certainement connues, la plupart ont été inventées par un grand esprit et viennent de très loin~! Tellement loin que la plupart sont annotées pour vous offrir une explication\footnote{Si vous avez recours aux explications, ne vous attendez pas à ce que le public soit plié de rire (je parle en connaissance de cause).}. \\ \\
	Trêve de plaisanterie et place aux fous rires~!
	
	\chapter{Questions}
	Des questions que tout le monde se pose, des interrogations existentielles.\\ Apportons de la lumière sur les questions que l'Humanité peut avoir~!
	\newpage
	\begin{enumerate}
		\item Où un vegan va-t-il jouer~? À Las Vegan~!
		\item Pourquoi un piano est-il très fort en cours~? Parce qu’il a de bonnes notes.
		\item Pourquoi dans le film Pinocchio, le héros a-t-il un nez aussi long~? Parce que c’est un film Disney\footnote{Dix nez.}.
		\item Qu’est-ce qu’un père qui veut devenir une mère~? Un trans parent.
		\item Quelle est la lettre la plus matinale~? $\tau$\footnote{Tout se fait tôt avec $\tau$.}.
		\item Comment appelle-t-on l’emploi du temps d’une bouteille d’eau~? L’aqua planning. 
		\item Comment une personne peut-elle s’immoler\footnote{Six mollets.} alors qu’elle n’a que deux mollets~?
		\item Quel est le plat cuisiné qui n’est jamais servi à l’heure~? Le steak tartare\footnote{Parce qu'il arrive tard-tard.}. 
		\item Quelle est la boisson préférée des carnets~? Le lait caillé\footnote{Le lait [des] cahiers.}.
		\item Pourquoi un gars de l'AFNOR rate-t-il toujours ses photos~? Parce qu'il croit que pour avoir une photo de qualité, il faut régler l'ISO\footnote{Deux significations, selon le domaine~: soit International Standard Office (pour les normes de tout genre, dont la norme donnant des directives sur la qualité~: ISO9001), soit c'est une quantification de la sensibilité du récepteur d'image de l'appareil photo (film ou capteur photographique). Donc en mélangeant les deux, une photo avec un ISO réglé à 9001 aura un grain non négligeable (de nuit) et n'aura peu de sens de jour.} à 9001. 
		\item Pourquoi ne faut-il pas énerver certains gobelets~? Parce qu’ils sont verre de rage\footnote{Et si t’as un verre vert en verre qui vient d’Anvers, c’est la totale~!}.
		\item Si un voyageur indélicat met une chanson assez médiocre à fond avec son enceinte portable, le volume sonore se quantifiera-il en Décibels ou en Décimoches\footnote{Le décibel est une échelle qui sert à mesurer l'intensité sonore d’un son. Mais si ce dernier n’est pas beau (féminin~: belle), est-ce que c’est moche~?}~?
		\item Qu’est-ce que cinq cents personnes qui n’arrivent pas à avoir pied à la plage~? Une ramette de papier\footnote{Ramette = un paquet de 500 feuilles (ou de \guillemotleft pas pied \guillemotright).}.
		\item Pourquoi le temps est une notion vraiment sournoise~? Parce qu’elle fait souvent des leurres\footnote{Et une montre sera son complice~: elle dit l'heure (dealer).}.
		\item Comment appelle-t-on la mise à l’écart d’un lézard~? La mise en tarentaine\footnote{Un mélange entre quarantaine et tarente.}.
		\item Pourquoi les bananes sont-elles aussi fines~? Car elles font beaucoup de régimes\footnote{Les régimes de bananes sont les sortes de grappes de bananes quand elles sont sur l'arbre.}~!
		\item Qu’est-ce qu’un livret d’instructions qui danse~? Un Manuel Valls\footnote{Et il maîtrise très bien la valse.}.
		\item Pourquoi ne faut-il pas regarder un film avec une aile d’avion~? Parce qu’elle peut te spoiler\footnote{Un spoiler (ou destructeur de portance) permet de réduire la portance de l'avion et de le freiner.}. \\\textit{\underline{Variante}~: Marche aussi pour les voitures de F1.}
		\item Pourquoi Sting sait-il très bien s’épiler~? Parce qu’il a la Police\footnote{Et la peau lisse du coup.}. 
		\item Pourquoi un moteur est-il vu comme un très bon magicien dans une voiture~? Parce qu'il peut faire plein de tours\footnote{Dont un super tour~: quand il tombe en panne, la facture du garagiste devient, comme par magie, très salée~!}.
		\item Pourquoi ne faut-il pas jouer avec l’eau à Salzbourg~? Parce que l’eau triche\footnote{L'Autriche.}~!
		\item Qu’est-ce qu’une algue qui est synchro avec la musique~? Une algue au rythme\footnote{Ou algorithme (il y a un algo pour bien danser apparemment)~!}.
		\item Avec quoi une fusée mange-t-elle sa soupe~? Avec une tuyère\footnote{Extrémité en forme d’hyperboloïde qui sert à expulser les gaz de la fusée.}.
		\item Est-ce que le père de Shrek était un pervers\footnote{Car son père est vert~!}~? \\\textit{\underline{Variante}~: avec Hulk.}
		\item Pourquoi un palmier a-t-il de vrais potes~? Parce qu’il n’a que des amis de longue datte\footnote{Les palmiers produisent des dattes.}~!
		\item Quel est le secteur le plus porteur~? Celui de la construction~: ils font des murs porteurs.
		\item Pourquoi Staline avait-il une humeur massacrante \footnote{Ce qui pourrait expliquer pourquoi il a tué autant de personnes.}~? Parce qu’il s’est levé du pied communiste\footnote{Le communisme est un parti d'extrême gauche. Donc il s'est levé du \textbf{pied} \textit{très} gauche. Et pour gérer son pays, il n'a pas levé le \textbf{pied}, a sûrement pris son pied, jusqu'à partir les \textbf{pieds} devant~!}.
		\item Pourquoi peut-on souvent voir des forces de l'ordre dans les salons de beauté ? Parce qu'ils ont la peau lisse\footnote{La police.}~!
		\item Pourquoi les cavistes sont-ils balèzes en cours~? Parce qu’ils ont plein de vin sur vin. 
		\item Pourquoi le Capitaine Crochet a-t-il participé à la Nouvelle Star~? Parce que c’est une émission de télé-crochet.
		\item Pourquoi peut-on dire que la Ferrari est une écurie~? Parce qu’elle a plus de trois cents chevaux\footnote{Des canassons ou des chevaux-vapeur~?}.
		\item Quel serait le comble pour les gens bons~? De payer des frais de port lors de leurs achats en ligne\footnote{Gens bons (jambon) / Frais de port (frais de porc).}.
		\item Si Freddie Mercury avait fait de la boxe, qu’est-ce qu’il serait devenu~? Un guet-apens\footnote{Gay tapant.}.
		\item Pourquoi une graine a-t-elle toujours mal à la tête~? Car elle a deux mi-graines\footnote{Deux mi-graines = une graine entière~!}.
		\item Pourquoi une personne qui porte un bonnet respire-t-elle mieux~? Parce qu’elle a deux nez\footnote{Un nez au milieu du visage et un beau nez sur la tête.}.
		%\item Comment reconnaît-on une orange qui prend tout son temps~? C’est celle qui n’est pas pressée\footnote{Mais si l’orange est pressée, elle doit être fatiguée et pour cause~: elle n’a plus de jus~!}. \\\textit{\underline{Variante}~: Marche aussi avec un citron.}
		\item Avec quoi les chefs de projets se lavent-ils~? Avec un GANTT\footnote{Un diagramme de GANTT est un outil de planification des tâches. Il est très utilisé lors de la réalisation d’un projet.} de toilette.  
		\item Qu’est-ce milkshake~? 20 carnets de 50 chèques\footnote{Ce qui fait un total de 1000 chèques.}.
		%\item Quel est le comble pour le PDG de Kia~? D’avoir comme téléphone un Nokia\footnote{Car c'est un No-Kia (pas de Kia)~!}.
		\item Pourquoi les petits pois sont-ils aussi fêtards~? Parce qu'ils sont souvent en boîte\footnote{Et vu qu’il fait noir dedans, c’est une boîte de nuit~!}~!
		\item Que font les lettres de l’alphabet grec lorsqu’elles ne se sentent pas bien dans leur peau~? Elles vont voir le $\Psi$\footnote{C'est la lettre \guillemotleft~Psi~\guillemotright.}.
		\item Que fait un mur lorsqu’un marteau lui porte trop préjudice~? Il porte plinthe\footnote{L’espèce de frise de carreaux en bas d’un mur. }.
		%\item Pourquoi les pneus sont-ils de gros délinquants en hiver~? Parce qu’ils sont enchaînés\footnote{À une époque, on enchaînait les prisonniers}. 
		\item Pourquoi ne faut-il jamais prendre de boissons gazeuses à la mode~? Parce que sinon, ça fait de la coca in\footnote{« \textit{in} » = à la mode. \textit{I'm out.}}~!
		\item Qu’est-ce qu’une gousse d’ail qui est malade et fatiguée~? Un aïoli\footnote{Un bon plat provençal.}.
		%\item Pourquoi la Belle au Bois Dormant est-elle née durant la meilleure époque~? Car elle a vécu pendant l’Âge d’Or\footnote{L’âge d'or est un mythe qui apparaît principalement dans la mythologie grecque puis la mythologie romaine (qui s'y réfère sous le nom de « règne de Saturne »). L’âge d'or fait partie du mythe des âges de l'humanité, avec l'âge d'argent, l'âge d'airain et l'âge de fer. [Extrait de Wikipédia]}.
		\item Quelle est la couleur la plus patiente~? Le jaune parce que Jonathan\footnote{Variante plus classe que le \guillemotleft~Qu'est-ce qui est jaune et qui attend~? Jonathan~!~\guillemotright}.
		\item Que fait un circuit électrique lorsqu’il va à la banque~? Il ouvre un compte courant\footnote{Un compte courant \textbf{électrique}.}.
		\item Quel est le jeu dans lequel les martiens excellent-ils~? Les arts martiaux\footnote{\underline{Conseil}~: il est déconseillé de se battre contre Curiosity.}.
		\item Quel serait le comble pour Gary Kasparov~? Échouer\footnote{Échouer et échec sont des mots très proches.} à une partie d'échecs. 
		\item Quel est l’hymne des accumulateurs déchargés~? « Allons enfants de la batteri-ie, le jour de charge est arrivé \textmusicalnote ».
		\item Que font deux personnes de petite taille lorsqu'elles se mettent ensemble~? Une nainfusion\footnote{Une infusion (sauf si ce n'est pas leur tasse de thé). Et un chercheur du CEA dira \textsuperscript{[réf. nécessaire]}~: «~et s’ils rompent, ils feront une nainfission~».}.
		\item T’as trois poussins mais tu n'en voulais que deux. Que fais-tu~? T’en pousses un.
		\item Que fait un stylo quand il s’inquiète~? Il se fait un sang d’encre.
		\item Quel est la fée qui fournit les denrées alimentaires à l’armée~? La fée des rations.
		\item Quel serait le comble pour une femmelette~? De se marier avec une omelette~!
		\item Pourquoi Hitler ne voulait-il pas essayer de nouvelles choses~? Parce que qui ne tente rien aryen\footnote{Le parti d'Hitler a beaucoup promu le principe de la \guillemotleft~race aryenne~\guillemotright}. \\\textit{\underline{Variante}~: Avec Himmler ou Goebbels.}
		\item Quel est le fruit qui suit toujours les dernières tendances~? Les dattes\footnote{Elles sont branchées.}.
		\item Comment appelle-t-on un char qui a bon goût~? Un charbon\footnote{C'est un char qui est bon~! J'ai honte d'avoir noté cette blague dans ce recueil...}.
		\item Pourquoi ne faut-il jamais manger un plat qui est en train de faire un film~? Parce que ça tourne~!
		\item Quel est le jeu de cartes préféré des hamsters~? Le Ham-tarot\footnote{Du célèbre hamster Hamtaro. Un tel jeu de cartes devrait créer \guillemotleft~de grandes aventures~!~\guillemotright}.
		\item Quel est la poupée la plus dangereuse\footnote{Après Klaus Barbie (réf. aux Inconnus).}? La Barbie Turique\footnote{Les barbituriques sont des médicaments servant à l’anesthésie, mais sont bourrés de mauvais effets secondaires (dépendance et mortel à haute dose par ex.). Un des « ingrédients » de l’injection létale.}. 
		\item Pourquoi faut-il aller en soirée avec un modo\footnote{Un modérateur (ou modo) est une personne chargée de modérer une communauté en ligne~: faire vivre la communauté, filtrer les messages illégaux, gérer l'infrastructure, etc.}~? Pour  boire avec Modération\footnote{Parce que l'abus d'alcool est dangereux pour la santé. À consommer avec modération.}. 
		\item Quel est le jeu favori d’un ver~? Le solitaire\footnote{Le ver solitaire est un véritable fléau.}.
		\item Pourquoi un manchot ne procrastine\footnote{Remettre les choses à demain.} jamais~? Parce qu'il ne remet jamais les choses à deux mains. 
		\item Si t'as un coup de blues, pourquoi faut-il avoir un compas sur soi~? Car il fait preuve de compassion.
		\item Pourquoi ne faut-il jamais acheter un téléphone à un avion~? Pour qu’il évite de décrocher\footnote{Lorsque l’avion « pique du nez ». De toute façon, depuis quand un avion a besoin d'un téléphone~?}.
		\item Pourquoi certains fabricants de chaussures sont-ils toujours déconnectés de la réalité~? Parce qu’ils sont à côté de leurs pompes.
		\item Dans quelle université un copieur va-t-il~? \\À la fac Simile\footnote{Un fac-simile est une reproduction d’un document historique. Un contrefacteur peut suivre les mêmes cours mais il aura mal tourné~!}.\\\textit{\underline{Variante}~: Marche aussi avec une photocopieuse (même si ça parait bizarre).}
		\item Quel est le comble pour une mante religieuse~? D’être athée.
		\item Pourquoi une feuille de papier n’aime pas se baigner~? Parce qu’elle n’a pas pied.
		\item Qu’est-ce qu’un ingénieur qui a de bonnes manières~? Un ingénieur poli technicien.
		\item Pourquoi est-il impossible de s’ennuyer lors d’un spectacle de tango~? Parce que c’est une performance très dense\footnote{Marche avec toutes les danses.}~!
		\item Quel est le vêtement le plus triste~? La blouse\footnote{Car elle a le blues. Marche mieux pour un chimiste.}.
		\item Quel serait le comble à propos du film \textit{Pirates des Caraïbes}~? Que l'acteur interprétant Jack Sparrow (Johnny Depp) pirate le film sur Internet\footnote{Au choix~: eMule, TPB, etc}. 
		\item Pourquoi un haut-parleur attend-il un enfant~? Parce qu’il est enceinte.
		\item Quel est l'exercice le plus difficile à finir~? L'exo mille\footnote{Lexomil (molécule active~: Bromazepam) est un médicament qui traite l'anxiété. Une utilisation incorrecte peut cependant entraîner une accoutumance et des effets secondaires tels qu'une réduction des fonctions moteurs et cognitives et surtout une fatigue chronique.}. 
		\item Pourquoi un sapin n’est-il jamais content en hiver~? Parce qu’on l’enguirlande\footnote{Après l'avoir coupé. En fait, il y a plusieurs raisons~!\\Et par conséquent, il a les boules.}.
		\item Si une personne parle du mont Everest, est-ce que c’est un haut-parleur\footnote{Parce qu'il parle d'en haut (haut parleur).}?
		\item Qu'est-ce qu'un chien qui a de la fièvre~? Un hot-dog. 
		\item Quel est le plat préféré du général de Gaulle~? Le plat de résistance. \\\textit{\underline{Variante}~: Marche aussi avec Jean Moulin, Antoine Simoni et de très nombreux résistants}.
		\item Pourquoi vendeur de disques est un métier à risques~? Parce qu’ils ont des CD\footnote{Ils ont DCD. RIP~!}.
		\item Pourquoi n’est-il pas bon d’être un glaçon~? Parce qu’il paie plein de frais\footnote{Vivement l'été pour en avoir moins (même si ce pauvre glaçon pourrait en faire les frais).}.
		\item Pourquoi faut-il faire de la collocation avec Napoléon~? Parce qu’il a un Bonaparte.
		\item Quels sont les noms qui peuvent provoquer de très gros débats~? Paul et Mique\footnote{Comptons sur eux pour faire polémique~!}.
		\item Pourquoi Mickey a-t-il une voix aussi aiguë~? Parce qu’il a joué au saute-mouton avec une licorne dans son enfance\footnote{D'où cette voix de castrat (littéralement).}.	
		\item Pourquoi le bois est-il connu pour être silencieux~? Parce qu’il sait stère\footnote{Une stère représente 1~m\textsuperscript{3} de bois (environ 500kg).}.
		\item Quelle est la voiture la mieux habillée~? La Polo \footnote{De Volkswagen.}.
		\item Pourquoi une personne qui quitte la Picardie ressort-elle plus en forme~? Parce qu'elle est sortie d'un Somme\footnote{C'est un département de la Picardie.}. 
		\item Pourquoi Spiderman est-il bon en peinture~? Parce qu’il sait faire de bonnes toiles\footnote{D'araignées~! \textit{Sa place est dans un musée~!}}.
		\item Quel est le comble pour une tablette de chocolat blanc~? D’être vendue au marché noir. 
		%\item Quel est le comble pour un vendeur de prêt-à-porter~? D’avoir une fringale\footnote{Car il vend des fringues.}.
		\item Quelle est la personne la plus âgée~? Pierre\footnote{Parce qu’il a l’âge de Pierre~!}.
		\item Que dit une lotion bas de gamme à une lotion de luxe (genre Chanel)~? «~Tu n’as pas la lotion de l’argent~!~»\footnote{Avoir la notion de l'argent. C'est sûr que pour acheter des lotions hors de prix, on perd cette notion.}
		\item Pourquoi un annuaire téléphonique présente-t-il une très grande habileté~? Car il a beaucoup d’adresse.
		\item Quel est le comble de M. Sanchez~? De ne trouver nulle part où s’asseoir\footnote{Surtout quand on a 100 chaises.}.
		\item Pourquoi faut-il donner des carottes à Diam’s~? Pour faire des carottes râpées. \\\textit{\underline{Variante}~: Marche avec tous les rappeurs (dont JUL).}
		\item Quelle est la friandise préférée d’une horloge~? Les Tic-Tac.
		\item Quel est le comble pour la marque Superdry~? De vendre des bouteilles d’eau\footnote{Surtout pour une marque qui se dit être \guillemotleft super sèche \guillemotright}. 
		\item Quel est l’arbre le plus frustré~? Le sapin~: il a les boules\footnote{Plus spécialement à l'approche de Noël.}.
		\item Pourquoi un pack de bouteilles restera immobile si on le met en plein milieu de la route~? Parce qu’il y a des bouchons. 
		\item Qu’est-ce qu’un train qui n’a jamais pris la pluie~? C’est un véhicule intrinsèque\footnote{«~Un train sec~».}.
		\item Quelles sont les chaussures préférées de la Faucheuse~? Les pompes funèbres.
		\item Quel est le comble pour un plombier~? D’être en manque de tuyaux. \\\textit{\underline{Variante}~: Il peut aussi prendre la fuite, ou bien ne pas savoir se canaliser.}
		\item Que se passe-t-il lorsque Batman\footnote{Batman ou batte-man~?} se prend une balle~? Il fait un home-run.
		\item Pourquoi il fait toujours chaud dans une voiture en été~? Parce qu’il y a un chauffeur.
		\item Qu’est-ce qu’un cochon avec un pinceau~? De lard.
		\item Quel type de culture (dans l’agriculture) est la plus drôle~? La rizière\footnote{Qu’est-ce qu’on riz~!}.
		\item Si John Lennon se fait mixer\footnote{Au mixeur ou au blender, pas par un DJ (quoique, ça marche avec les deux en même temps).}, est-ce que ça va faire du Beetlejuice\footnote{C’est le nom d’un film… ou d’un jus de Beatles (inventé). Cependant, je n'aurais pas envie d'en boire.}~? \\\textit{\underline{Variante}~: avec Paul McCartney ou Ringo Starr ou George Harrison}.
		\item Quelle est la lettre la plus fraîche de l’alphabet dans le Nord~? La lettre B~: parce qu’il y a plein de beffrois là-bas~!
		%\item Dans un jeu télévisé, pourquoi certaines réponses sont en rouge~? Parce que l’animateur demande de «~Taper votre réponse~».
		\item Qu’est-ce qu’un mauvais train~?  Un train qui arrive sans crier gare.
		\item Quel est le comble pour des jumelles~? D’être myope\footnote{Les jumelles sont capables de voir de loin. Contrairement aux myopes.}.
		\item Quel est le meilleur ami de la farine~? C’est l’ami Don.
		\item Que fait un clown dans un casino~? Il joue au Blague Jack\footnote{Et s’il en raconte plus de 21, il est cramé~!} !
		\item Pourquoi ne faut-il jamais aller à un rencart avec un cuniculiculteur\footnote{(Mot compte triple.) C'est un éleveur de lapins.} ? Parce qu'il peut te poser un lapin. 
		\item Avec quoi George W. Bush se lave-t-il~? Avec du bain de bouche\footnote{Du bain de Bush.}.  
		\item Pourquoi une poupée vaudou est-elle bien habillée~? Parce qu’elle est tirée sur quatre épingles~!
		\item Pourquoi les vampires préfèrent sucer le sang des chevaux~? Parce qu’ils ont plus de chance d’avoir du pur sang. 
		\item Quel est le constructeur aéronautique qui sait le mieux rebondir après une crise~? Bo(e)ing\footnote{Sinon, le cours de l'action se crashe.}.
		\item Si un individu est, étymologiquement, un être indivisible, est-ce que Louis XVI est un individu~?
		\item Quel est le comble pour une personne qui vit à Fos-sur-Mer~? D'être sceptique\footnote{Sceptique en lisant cette blague, oui. Mais ça fait aussi fosse sceptique.}.
		\item Quel est le fruit préféré des végétariens ? La « pas steak\footnote{La pastèque (et le fait de ne pas manger de steak). Dans les deux cas, c'est une affaire juteuse !} ».
		\item Quelle est la bière préférée des littéraires ? La rousse\footnote{Comme le dictionnaire Larousse.}.
		\item Pourquoi Van Gogh dormait mal vers la fin de sa vie ? Parce qu'il ne dormait plus sur ses deux oreilles\footnote{Surtout quand il s'est coupé une oreille.} !
		\\\textit{\underline{Variante}~: Pourquoi Van Gogh ne faisait pas attention à ce que les gens lui disaient ? Parce qu'il ne les écoutait que d'une oreille\footnote{À partir de là, difficile de lui faire entendre raison.}.}
	\end{enumerate}
	
	\chapter{Blagues de geek}
	Une rubrique pour ceux qui pensent qu'il y a 1024 mètres dans un kilomètre\footnote{Et un encore meilleur geek est celui qui parlerait plutôt de kibimètre.}. 
	\newpage
	\begin{enumerate}
		
		\item \texttt{Quelle est l’arme préférée des navigateurs Web~? Le lance-requêtes.
		\item Pourquoi Larry Page a-t-il des goûts très prononcés~? Parce que ses Google\footnote{Ses goûts gueulent~! Pourvu que ça ne soit pas une blague de mauvais \textit{goût}~!}~! 
		\item Que dit une disquette qui est trop pleine~? \guillemotleft~Faut que je perde quelques kilos\footnote{Kilooctets, bien sûr~!}~\guillemotright~!
		\item Quel est le logiciel phare pour le goûter ? C’est Firefox\footnote{Marche aussi avec Opera, Google Chrome, Konqueror, Maxton, Internet Explorer, Safari, Mosaic, Netscape, etc. \\Avec n’importe quel navigateur Internet, en somme~!}~: il a de si bons cookies\footnote{Les cookies sont ces fichiers qui sont créés par le site Internet pour lire/stocker certaines informations de l’utilisateur pour « améliorer l’expérience utilisateur » lors de la visite d’un site.}~!
		\item Lorsqu’il mange, sur quoi un geek mange-t-il~? Sur une nappe IDE\footnote{C’est ce qui relie les disques durs/lecteurs CD entre eux dans un (relativement vieil) ordinateur.}.
		\item Que faut-il donner à un navigateur Internet lorsqu'il a faim~? Des cookies\footnote{C’est un fichier local dans lequel certaines données liées à un utilisateur sont stockées (exemple d'utilisation~: \sout{les pubs ciblées} pouvoir se connecter automatiquement à un site sans avoir à taper son mot de passe).}.
		\item Lorsqu'un geek a une déconvenue, ça le fichier\footnote{On peut dire que ça le fait chier aussi mais restons polis~!}.
		\item En été, quelle est la touche la plus utilisée~? F5\footnote{C'est la touche qui  $\textbf{rafraîchit}$ une page.}.
		\item Un geek azerty en vaut deux\footnote{Parce qu'un geek averti en vaut deux aussi.}.
		\\\textit{\underline{Variante}~: Avec un clavier.}
		\item Quel est le navigateur préféré d'un cuisinier~? \\Mozzarella Firefox.
		\item Comment un terminal Linux fait-il pour aller quelque part~? Il se met en root.
		\item Que fait un site Internet quand il vaut voyager~? Il fait ses balises\footnote{Une personne normale ferait ses valises.}~!
		\item Un informaticien peut aussi faire des greffes de cœur~: lorsqu’il change de processeur\footnote{Un processeur contient un ou plusieurs cœurs. D’où la \textit{punchline} de fou~!}.
		%\item Un chien programmeur ne cherche pas d’os~: il compile son propre os\footnote{Compiler son OS \textit{(Operating System)} permet de personnaliser son OS et de rajouter/supprimer des composants qui ne seront pas utiles. En général, c'est une pratique réservée aux utilisateurs chevronnés.}.
		\item Quel est le plat favori des processeurs~? Les pâtes thermiques\footnote{La pâte thermique permet un refroidissement plus efficace d’un processeur en favorisant les échanges thermiques avec le radiateur (et donc avec l’extérieur).}.
		\item Comment se porte une somme de contrôle qui a raté~? Elle a le check seum\footnote{Le checksum est une somme de contrôle (en gros, pour vérifier qu’un fichier/message n’a pas été changé en cours de route).}.
		\item Pourquoi un pirate informatique aime-t-il son travail~? Parce que ça lui tient hacker.
		\item Quelle est la chaîne de montagnes préférée de la mémoire~? La Cordillère des NAND\footnote{Les portes NAND sont les principaux constituants de la mémoire flash (RAM ou SSD)}.
		\item Qu’est-ce qu’une boucle for écrite il y a quelques centaines de lignes~? Une boucle fort fort lointaine\footnote{Ça, c’est quand Shrek fait de la programmation dans son royaume.}.
		\item Comment Al Gore se reproduit-il~? Il lance Maple et il fait un algorithme\footnote{Al Gore ithme. Hahahaha!}.
		\item Pourquoi les flaques d’eau ont une bonne qualité sonore? T’as déjà vu une MP3 d’eau~?\footnote{Le FLAC est un format de musique avec une très bonne qualité contrairement au MP3 qui a une qualité moyenne.}
		\item Deux clés USB discutent~:
		\begin{itemize}
			\item[-] J’me sens FAT\footnote{Un système de fichiers, utilisé pour les cartes SD et clefs USB. Ici, c'est une manière d'exprimer sa \textbf{\underline{fat}}igue.}\ldots
			\item[-] Ah moi, je me sens NTFS\footnote{Un système de fichiers, pour les disques durs et les clefs USB de grande capacité.}~!
		\end{itemize}
		\item Quelle est la drogue préférée des signaux numériques~? Le LSB\footnote{LSB pour Least Significant Byte (ou bien lsb pour Least Significant Bit). \\En informatique, c’est le bit/octet de droite.}.
		\item Ce sont deux logiciels qui sont en train de s’installer. L’un deux s’installe sur le D:\textbackslash. Du coup, l’autre logiciel lui dit~: « Allez, viens danser\footnote{Dans C:\textbackslash}~! ».
		\item Quel est le type de téléphone favori des nudistes~? Les téléphones nus\footnote{Les téléphones vendus sans forfait (et donc payés au prix fort)}.
		\item Que fait un voleur d’ordinateurs après son larcin~? Il prend la F8.
		\item Si on pêche une raie et puis qu’on la plonge dans un seau de peinture bleue, ça deviendra un blu-ray~?
		\item Pourquoi Staline était-il au courant de toutes les actualités de son pays~? Parce qu’il suivait les fl-URSS\footnote{Les flux RSS étaient utilisés pour recevoir les informations de certains sites)}.
		\item Quel est le groupe sanguin d’un geek~? C++\footnote{Le C++ est un langage de programmation.}.
		\item Avec quoi un utilisateur de Linux recouvre sa piscine~? Avec une Bash\footnote{C'est le nom d'un interpréteur de commandes sous Unix/Linux.}.
		\item Si un programmeur était un Dom Juan, que maîtriserait-il le plus~? Le drag and drop\footnote{Le « drag and drop » (glisser-déposer en bon français) est une manière de déplacer des éléments dans une interface graphique.\\ Mais pour un Dom Juan (quelqu'un qui aime bien les conquêtes éphémères), d’abord il « drague » ses partenaires et ensuite il les « drop » (les laisse tomber).}. 
		\item Pourquoi les Intel Xeon sont-ils de grands \textit{lovers}~? Parce qu'ils ont douze cœurs à prendre\footnote{Nous parlons ici de la gamme de processeurs Xeon qui peut avoir 12 cœurs (voire plus maintenant).}~!
		\item Si Twitter n'était utilisé que par des dev en C/C++, quel serait le \textit{trend topic} indétrônable~? \#include\footnote{C'est utilisé pour inclure des librairies (donc incontournable pour un programme en C)}.	
		\item Comment les musiciens font-ils pour sauvegarder leurs machines~? En faisant un Bach-up\footnote{Référence à Jean-Sébastien Bach. Et un \textit{backup}, c'est une sauvegarde dans le jargon.}.
		\item Tu sais pourquoi ML\footnote{Un acronyme pour un truc de ton choix. Un exemple~: Mon Livredeblagues} se vend bien~? Parce que tous les sites veulent HTML\footnote{« Veulent Acheter ML ». Le HTML est un langage utilisé pour les pages Web.}.
		\item Que porte un individu qui n’a pas beaucoup de \textit{string}~? Un \textit{char}\footnote{Un \textit{string} est une chaîne de caractères qui contient un ou plusieurs \textit{char}.}.
		\item Pourquoi un disque dur est-il souvent diabolique~? Parce qu’il est SATAnique\footnote{Référence avec la connectique SATA qui est utilisée dans les disques durs (récents)}.
		\item Deux ordinateurs discutent. L’un d’eux dit~: « Mec, j’ai mal au DOS\footnote{DOS = Disk Operating System, un vieil OS à ligne de commandes by Microsoft}\ldots »
		\item Il y a 10 sortes de gens dans le monde~: ceux qui comprennent le binaire et ceux qui ne le comprennent pas.
		\item Quel est le sandwich favori des éditeurs de photos~? Le Crop Monsieur\footnote{\textit{To crop}, c'est rogner une photo. Et inutile de rappeler ce qu'est un croque-monsieur~!}~!
		\item Pourquoi les supercalculateurs sont-ils nuls pour raconter des blagues~? Parce qu’ils font beaucoup\footnote{Quelques milliards parfois~!} de flops\footnote{Un FLOP (FLoating-point Operation Per Second) quantifie le nombre d’opérations (à virgule flottante) que peut faire une machine.}. 
		\item Qu’y a-t-il au pied d’un arc-en-ciel~? \\Un Nyan Cat qui s’est écrasé au sol.
		\item Qu'offre Amazon pour Noël~? Un Kindle Surprise\footnote{Sponsorisé (bénévolement) par Ferrero\textregistered}.
		\item Sur quoi un ingénieur se repose-t-il~? \\Sur un Matlab\footnote{Logiciel très puissant de calcul et de simulation. La (n-ième) main droite de l’ingénieur.}.
		\item Qui est le meilleur programmeur de batch\footnote{Les fichiers *.bat}~? Batman.
		\\\textit{\underline{Variante}~: Marche aussi avec un smartphone.} 
		\item Qu’est-ce qu’un T-Shirt Nike dans la mer~? Un élément à virgule flottante\footnote{Les nombres déclarés comme $\textit{float variable}$. Ils sont utilisés en programmation pour les nombres décimaux.}.
		\item Un geek n’achète pas de canapé convertible mais un canapé qwertyble.
		\item Pourquoi un geek peut-il être considéré comme un meurtrier~? Parce qu’il ne fait qu’exécuter\footnote{Exécuter signifie soit lancer un programme, soit tuer quelqu'un.} ses programmes.
		\item Pourquoi les gens trouvent qu'un tableau Excel prend trop de place~? \\Parce que c'est un adepte du \textit{manspreadsheet}\footnote{Le \textit{manspreading} est une posture décriée~: en gros, c'est lorsqu'une personne écarte ses jambes en s'asseyant dans un lieu public, gênant ses voisins. Ici, c'est plutôt autre chose~: remarquons la similitude entre \textit{spreading} et \textit{spreadsheet} (feuille de calcul).}. 
		\item Que met un programme comme sous-vêtements~? Il met des <string.h>\footnote{Librairie incontournable lorsqu'on traite du texte en C.}.
		\item Qu’est-ce qu’un DVD mort~? Un DVD-Rip.
		\item Un programmeur ne prend pas de vitamine C, il prend de la vitamine C++.
		\item Quelle est l'interface de programmation utilisée par McDo? L'API Meal\footnote{Une API (Application Programming Interface) est une interface entre différentes parties d'un programme.}.
		\item Jick et Jack, une paire d’écouteurs, sont en train de braquer une banque\footnote{Dans ces blagues, tout arrive~!}. Déplorant la faible réactivité de son acolyte, Jick s’écrie alors~: «~Mais qu’est-ce que tu fiches, Jack\footnote{Une fiche Jack~!}~?!~»
		\item Quelle est la boisson la plus alcoolisée selon un geek~? Alcohol~120\%\footnote{C’est un logiciel pour graver des CD ou pour créer un lecteur CD virtuel. \\Bref, le meilleur ami des fichiers ISO}.
		\item Deux programmeurs discutent au bar, pendant la soirée du 31 décembre~:
		\begin{itemize}
			\item[-] Quelle est ta résolution pour cette année~?
			\item[-] 1024x768. Et toi~?
		\end{itemize}
		\item Qu'est-ce qu'un torrent téléchargé par Hitler? Un génoseed\footnote{Un seed dans les torrents, c’est quelqu’un qui télécharge le torrent en question.}.
		\item Pourquoi est-ce une bonne idée de lancer le CD de Windows Vista comme un frisbee à son chien~? Parce qu’un chien aime les OS.
		\item Quel est le registre littéraire du code source de Windows~? Regedit\footnote{C'est l'application qui permet de modifier les clés du registre de Windows}.
		\item \guillemotleft~Un geek, au petit déj, il mange des serials, fume du crack pour être dans les clouds mais comme ce n’est pas bon pour la santé il met des patchs pour arrêter de fumer.~
		\item Un développeur alcoolique ne fait pas de \textit{merge request}\footnote{Dans un projet (qui utilise Git comme gestionnaire de versions), lorsqu'un développeur souhaite soumettre des \sout{bogues} améliorations, il peut demander à ce que ses changements soient intégrés et fusionnés dans le projet principal. Cela amène à une vérification pour voir notamment qu'il n'y a aucun problème causé par ce nouveau code.} : il fait une \textit{murge\footnote{Se murger, c'est boire de manière déraisonnée.} request}.
		\item Quel est le comble pour un boulanger Indien quand il écrit des algos ? Avoir des NaN\footnote{Le naan est un pain indien (délicieux). Un NaN (\textit{not a number}), c'est une valeur retournée lors d'une opération illégale (arcsinus d'un nombre supérieur à 1, log d'un nombre négatif, etc)}.
		%\item...
		% add new jokes related to geeks here (before closing bracket for monospace font)
		}
	\end{enumerate}
	
	\chapter{\textit{In English!}}
	Jokes do not have borders. To make everyone laugh, we can not only tell jokes in French: most people can not understand them. That's why, this chapter exists. 
	\\Just don't pay attention to my French accent!
	\newpage
	\begin{enumerate}
		\item Why does the population decrease in Turkey during the Thanksgiving? Because during this day, Americans eat turkey.
		\item Why Toto’s CD-ROM are bent, as if they were set on fire? Because he burns them\footnote{To burn has two meanings: set something on fire or write data on a disk. Toto chose the wrong meaning. Too bad.}. 
		\item What would be the worst thing for a screensaver? It’d be to let a screen die\footnote{Because the screensaver was unable to save the screen. It had one job.}. 
		\item What's the favorite George Sand activity? Making sandcastles.
		\item How would we call Iron Man if he was killed? Death Metal. 
		\item What do you do when you are Hungary? You eat Turkey.
		\item Why is a computer loved? Because it has many fans\footnote{A fan cools down a room/piece. An other meaning is a person who loves somebody/something.}.
		\item According to a mathematician, what is the best syrup? The Maple syrup\footnote{Maple is a symbolic and numeric computing environment.}.
		\item This is the story of a vector who wants to beg forgiveness. So it goes in a church and explains its problem to a priest: “My Father. I wanted to be expressed in another basis but I think I committed a sin\footnote{Here, the sin can either refer to a misdeed or to the trigonometry function $sin(x)$.}”.
		%\item Why is Maped\footnote{A famous brand of rulers (and other school supplies).} the king of the planet? Because it rules the world.
		\item What is the least expensive animal? It is the ram because it is sheep.
		\item A mathematician does not read comics. Instead, he reads conics\footnote{It is a curve (more precisely: the intersection of a cone with a plane).}.
		\item Physicists do not think: they Fick\footnote{Fick’s Laws are used to describe diffusion. Here is the Fick Law seen in school: $J = -D\dfrac{\partial \Phi}{\partial x}$.}.
		\item Why should you never talk about abstract things to a builder? Because he prefers concrete\footnote{A really famous building material.} topics.
		\item Why should you never let buckets lying on the floor? Because someone could kick the bucket\footnote{Expression which means “to die”.}.
		\item Where is Wiener? Wiener is in the Khintchine\footnote{It is a reference to the Wiener-Khintchine theorem which states that the autocorrelation function of a wide-sense-stationary random process has a spectral decomposition given by the power spectrum of that process.}.
		\item What is an Australian with a little reach? An Aussie low scope\footnote{An oscilloscope. Here, we love scientific jokes!}.
		\item Why is it impossible for Marie Antoinette to win a race? Because she can not make a head start\footnote{She was beheaded in 1793.}!
		\item Why is a hand always in a hurry? Because it does not want to waist its time. 
		\item Who was the best bodybuilder\footnote{Someone who makes a lot of sports to develop one's musculature.}? Frankenstein\footnote{He literally built a body, even if it was known as the "monster of Frankenstein".}. 
		\item Sometimes, I love talking with boring tools\footnote{Things that enlarges a hole in manufacturing}. After a certain moment, we can have deep conversations. 
	\end{enumerate}
	
	\chapter{Blagues scientifiques}
	Rendons les cours de maths plus \textit{funs}~! 
	\\As we say in English: let's put the \textit{fun} back in \textit{mathematical \underline{fun}ctions}!  
	\\La chute de ces blagues a bien été \textbf{calculée}~!  
	\newpage
	\begin{enumerate}
		\item Qu’est-ce qu’un chauffeur qui conduit mal~? Un isolant\footnote{C’est pourquoi la laine de verre n’aura \textbf{jamais} de permis de conduire. Nous voilà sauvés~!}.
		\item Quel est le point commun entre les bébés et l’argon~? Leurs couches\footnote{C'est juste la marque qui change~: Pampers$ ^{\circledR}$ pour les uns et 1s, 2s, 2p, 3s, \ldots pour les autres.} sont toujours pleines.	
		\\\textit{\underline{Variante}~: Marche aussi avec le néon, l’hélium, le krypton, etc.}
		\item Un circuit logique ne se perd jamais~: il ne perd pas le NOR\footnote{Type de porte logique.}~!
		\item Qu’est-ce qu’un iPhone~? Un téléphone complexe\footnote{Avec $i \in \mathbb{C}$.}.
		\item Quelle est la solution de l’équation $e^x = segara$~? $ln(segara)$\footnote{Hélène Ségara~?}.
		\item Quel est l’hymne préféré des équas différentielles~? L’ODE\footnote{ODE pour Ordinary Differential Equation (ou équation différentielles en français). Celui qui dit que cette blague est ODE-ieuse sorte.} à la joie.
		\item Pourquoi un donut n’a jamais raison~? Parce qu’il a tore\footnote{La forme d’un donut (le tore, c’est comme un anneau en 3D).}.
		\item Les couples~: au début, ça commence avec un Bezout puis ça finit avec un Gauss\footnote{Et ça Cauchy ensemble en s’envoyant Euler\ldots}.
		\item Quel est le comble pour un vecteur~? De ne pas être aux normes. 
		\item Conseil~: pour Pile ou Face, jouez avec Alessandro Volta~: vous gagnerez à coup sûr\footnote{Bah oui, il fait toujours pile~!}~!
		\item Quel est le nombre qui est le plus fort à cache-cache~? La constante de Planck\footnote{C'est la constante toute petite ($h = 6.63 \times 10^{-34} J\cdot s$).}.
		\item Quel était le type le plus stressé au monde~? Pascal~: il avait la pression\footnote{Le Pascal est une unité de pression du SI (C'est la force d'un Newton sur une surface d'un mètre carré.)}~!
		\item Les mathématiciens ont aussi une \textit{happy hour}: 3:14. 
		\item Si un gars a son permis, comment pourrait-il être appelé~? Un conducteur ohmique\footnote{C'est pour ça que je n'ai jamais vu du plastique au volant d'une voiture.}. 
		\item Quelle est la matrice la plus laide~? La matrice identité\footnote{C’est la matrice $\mathbf{I_{2}} =  				
			\begin{bmatrix}
			1 & 0 \\0 & 1
			\end{bmatrix}$
			qui appartient à $\mathcal{M}_{2}(\mathbb{K})$.} $\mathbf{I_{2}}$.
		\item Quel type de foulard un vecteur porte-t-il~? Un Chasles\footnote{Référence à la relation de Chasles.}. 
		\item Pourquoi la fonction cosinus est têtue comme une mule~? Parce qu’elle est bornée\footnote{En effet~: $\forall x \in \mathbb{R}, \cos(x) \in [-1;1].$}.
		\item Que dit un mathématicien quand il se noie~? $Log$ $log$ $log$\ldots
		\item Deux personnes discutent à l’arrêt de bus~: \guillemotleft~Ce gars, il démarre au $\pi/2$~!\footnote{$\pi/2~rad$ = un quart de tour (ou 90\textdegree).}~\guillemotright. 
		\item Si Thomas Edison aimer porter des DC Shoes, est-ce que Nikola Tesla préférait les AC Shoes\footnote{T. Edison était celui qui promouvait le courant continu (\textit{Direct Current}, DC) alors que N. Tesla montrait les avantages du courant alternatif (AC), très utilisé aujourd'hui (car moins de pertes d'énergie sur de longues distances, facile à transporter, etc).}~?
		\item Les aquariums ont aussi leur code civil~: les occupants doivent obéir à la loi de Poisson\footnote{En proba~: la loi de Poisson est $\mathcal{P}(k) = P(X = k) = \frac{\lambda^{k}}{k!}e^{-\lambda}$}.
		\item Qu’est-ce qu’un filtre en colère~? Un filtre Wiener\footnote{Véner et Wiener se prononcent presque de la même manière~!}.
		\item En chimie, quelle est la représentation la plus \textit{hot}~? La représentation de Cram.
		\item Qu’est ce que 3.14 coloscopes~? Une coloscopie\footnote{3.14 coloscopes = $\pi$ coloscope = coloscop pi = coloscopie. \\Un coloscope est un planning d’interros orales en prépa (khôlles).}.
		\item C’est un couple de fonctions qui sont\ldots assez carrés. En voyant leur fiston, Nome, assez turbulent, les parents lui disent~: \guillemotleft~Oooh, sois poli, Nome~!~\guillemotright.
		\item Je voudrais appeler mon lapin Faraday. Comme ça, il sera dans une cage de Faraday.
		\item Quel est l’instrument de musique de prédilection des fonctions\footnote{Une fonction périodique peut être décomposée en harmoniques dans un spectre.} périodiques~? L’harmonica.
		\item Dans quel cinéma un physicien a-t-il l’habitude de regarder un film~? Au cinéma Tique\footnote{Ils font de super belles cinématiques.}.
		\item Quelle est la fonction la plus perdue~? La fonction logarithme népérien~: car $\ln$ s’égara.
		\item Comment une famille de vecteurs fait-elle pour avoir des enfants~? Elle utilise le théorème de la base incomplète\footnote{Soit une famille libre $(u_{1}, u_{2}, \ldots, u_{p})$ dans un espace vectoriel $\mathsf{E}$. Alors il existe $(u_{p+1}, u_{p+2}, \ldots, u_{n})$ tel que $(u_{1}, u_{2}, \ldots, u_{n})$ soit une base de $\mathsf{E}$. \\Ça, c’est le TBI. La meilleure manière d’agrandir une famille (de vecteurs). \\(Quoi, elle n’est pas marrante la blague~? Je crois qu’on est parti sur une trèèèès mauvaise série~!)}.
		\item Un mathématicien ne s’éclaire pas avec une lampe mais avec un projecteur orthogonal.
		\item Une calculette bon marché et un nombre complexe discutent. Le nombre complexe~: \guillemotleft~ Pourquoi tu ne me calcules jamais~?! Je suis là quoi\footnote{Une calculette simple ne peut pas faire de calculs avec des nombres complexes.}~!~\guillemotright.
		\item Pourquoi $e^{2i\pi}$ est-il un mauvais avocat~? Car il n’a que des arguments nuls\footnote{$Arg(e^{2i\pi}) = 0$.}. 
		%\item Pourquoi une poire est un fruit injectif~? Parce que son noyau est nul\footnote{$Ker(Poire) = 0$.}.
		\item Les formules en optique, c’est compliqué. Y’a pas photon\footnote{Détournement de l’expression « y’a pas photo ».}~!
		\item C’est une fonction non continue (ex~: $\tan(x)$) qui est nouvelle dans sa classe. Sauf qu’elle est rejetée. Du coup, le prof lui dit~: 
		\begin{itemize}
			\item[-] Tu sais, il faut que tu t’intègres dans ta classe.
			\item[-] Justement, je ne peux pas\footnote{On peut intégrer une fonction si elle est continue sur un intervalle donné.}.
		\end{itemize}
		\item Est-ce que donner à manger de l’Uranium 235 à un oiseau, c’est lui faire la Becquerel\footnote{La béquée et le Becquerel, deux termes si proches}~? \\\textit{Variante~: Un oiseau, à Tchernobyl, ne fait pas la béquée mais la Becquerel}.
		\item Si une personne se brûle au troisième degré et dérive sa brûlure\footnote{Avant que votre prof de maths fasse un infarctus, assurez-vous que « brûlure » soit dérivable (ou mieux~: un polynôme).}, est-ce que ça devient une brûlure du second degré~?
		\item  Pour le jeudi de l’Ascension, un électronicien fait le pont\ldots de diodes\footnote{Circuit électrique permettant de redresser une tension (une partie de la transformation AC $\rightarrow$ DC).}!
		\item Quel est l’espace vectoriel préféré des charcutiers~? Le $SO(6)$.\footnote{L’espace vectoriel « spécial orthogonal » des matrices carrées de taille~6 (sous-espace de $O(6)$).}
		\item \guillemotleft~ Riemann et les Laplaciens Crétins ~\guillemotright\footnote{Ubisoft se battrait pour obtenir les droits sur ce super titre de jeu.}.
		\item Que se passe-t-il lorsqu’on dérive une sinusite~? On obtient une cosinusite\footnote{Parce que $cos'(x) = sin(x)$. As English people say~: \textit{"It is a sin to make such a lousy joke"}.}.
		\item Alessandro Volta était un mec qui avait du potentiel\footnote{Il a inventé la pile électrique. Enfant, il devait être une vraie pile électrique~!}~! 
		\item Quel est le vêtement privilégié de i~? Le corset\footnote{Le corps $\mathbb{C}$.}.
		\item Quels sont les théorèmes les plus explosifs en sciences de l’ingénieur~? Les théorèmes généraux de la dynamite\footnote{Ils n’existent pas. Par contre, les théorèmes de la dynamique, c’est trop de la bombe~!}.
		\item \guillemotleft~L’électrostatique, c’est un truc de Gauss~!~\guillemotright
		\item Quel est le segment le plus vieux dans un repère~? Le segment $[AG]$.
		\item Quel est l’espace vectoriel le plus explosif~? C’est l’espace vectoriel $\mathbb{C}_{4}$.
		\item \guillemotleft~ Les schémas équivalents en élec, c’est pas ma tasse de Thévenin\footnote{Nom d’un schéma équivalent en élec.}~! ~\guillemotright
		%\item La meilleure déclaration d’amour~: \guillemotleft~Je te rapport de transformation\footnote{Parce que c'est noté $m$. À ne pas dire lors d'un premier date (après, fais comme tu veux).}~\guillemotright.
		\item Ce sont deux lames de fer qui voient passer une lame de cuivre. Alors la première lame dit à son acolyte~: \guillemotleft~Tu ne trouves pas qu’elle a un beau Cu~?~\guillemotright.
		\item Comment les atomes font-ils pour voir s’il va y avoir des bouchons dans la matière~? Ils regardent le Boson Futé\footnote{Un boson est une catégorie de particules (comme les photons).}.
		\item Les séries, quand elles rigolent, elles ont un Fourier\footnote{Ou un fou rire, comme ce que tu es en train d'avoir en lisant cette blague de qualité.}.
		\item Quel est la conjugaison du verbe chanter~? C’est $\overline{chanter}$. \\\textit{\underline{Variante}~: marche avec tous les verbes, même les plus insensés~!}
		\item Deux électrons sur le Titanic. Lorsqu’ils apprennent le naufrage imminent de leur navire, ils se mettent à crier \guillemotleft~AAAAAH~!!! Nous Coulomb\footnote{Une unité de mesure de charges. Et les électrons ont une charge~!}~!!! ~\guillemotright
		\item Pourquoi est-il difficile de faire de l’humour avec un opérateur différentiel~? Parce qu’il prendra au sérieux toute remarque au second degré\footnote{Puisque ça deviendra du premier degré, après dérivation.}.
		\item Pourquoi un mathématicien a-t-il toujours des techniciens de surface avec lui~? Pour pouvoir trouver les espaces propres\footnote{Un espace propre associé à une valeur propre est l'ensemble des vecteurs propres qui ont une même valeur propre et le vecteur nul.} plus facilement.
		\item Pourquoi $x^{2}+y^{2} < 1$ travaille 24h/24 et 7j/7~? Parce que c’est toujours [un] ouvert\footnote{Un ouvert est un ensemble mathématique.}~!
		% \item Que fait une application pour planer~? Il se fait une injection\footnote{Une application est injective si elle admet au plus un antécédent.}.
		\item Quelle est la fonction la plus confortable/douillette~? $\cos(y)$\footnote{Oui, cosy. Pour un côté mystère on peut aussi répondre par $\frac{d(\sin({y}))}{dy}$. \\NB: Remplacer \textit{y} par \textit{i} si c’est fait à l’oral.}. 
		\item Un chimiste n’est jamais surpris~: il cétonne\footnote{Un composé chimique qui s’écrit \chemfig{R-[0]C(-[2]R')(=[0]O)}.}.
		% \item On parle toujours de torseurs. Mais où sont leurs torfrères~?
		\item Un repère ne s’ennuie pas~: il tourne en ROND\footnote{Pour Repère OrthoNormal Direct}.
		\item \guillemotleft~L'électromagnétisme c'est un truc de Gauss, même si cela implique un flux d'informations conséquent. Néanmoins vu la densité volumique de formules et la charge de travail en amont, on ne risque pas de diverger. Oui, je sais, ce n’est pas le moment de faire des blagues bidons sur ça, d'autant plus que je ne suis pas comme une équation de Poisson dans l'eau. Bref, faut que j'aille prendre Euler...~\guillemotright
		\item Quel est le comble pour une matrice triangulaire inférieure~? Avoir un complexe d’infériorité\footnote{Comme une matrice triangulaire inférieure.}.
		\item Quelle est la matrice qui n’aura pas droit à des cadeaux pour Noël~? La matrice de passage\footnote{Parce qu’elle n’est pas sage~!}.
		\item Quel serait le comble de la fonction cosinus~? D’avoir une sinusite.
		\item Ce sont deux $\mathbb{R}$-espaces vectoriels qui se marient. Quelle alliance ont-ils choisi pour cet évènement~? Un anneau commutatif\footnote{Un ensemble dans lequel la multiplication est commutative (une condition nécessaire pour avoir un $\mathbb{R}$-espace vectoriel).}. 
		\item Qu’est-ce qu’un moteur qui ne fournit aucun effort~? C’est un moteur célibataire\footnote{Parce qu'il n'a pas de couple. Ha~!}.
		\item Pourquoi les calculettes en mode degré sont-elles plus populaires que celles en radian~? Parce que ces dernières sont en rad\footnote{Être en rade signifie « être abandonné »~: c’est si triste pour ces calculettes...}~!
		\item \guillemotleft~ L'été est fini~! Il faut ranger ses vêtements différentiels\footnote{En l'occurrence, ses vêtements $\partial t$.}~\guillemotright.
		\item Si Clara Morgane\footnote{C’était une star du X (cela dit, ça peut marcher avec les polytechniciens aussi -- l’X).} rayonne, fera-t-elle des rayons X~?
		\item Que dit un 0 quand il voit un 8~? \guillemotleft~Quelle belle ceinture~!~\guillemotright.
		\item Pourquoi il faut toujours miser sur les numéros 2, 3, 5, 7 et 11 au PMU~? Parce qu’ils sont toujours premiers.
		\item Si une bobine s'inscrivait à la Nouvelle Star, quelle chanson chanterait-elle~? Daniel Balavoine~: \textit{le Chanteur}\footnote{La chanson qui commence par \guillemotleft~Je me présente, je m'appelle Henri \textmusicalnote~\guillemotright.}.
		\item Pourquoi un jardinier est-il un bon mécanicien~? Parce qu’il a plein d’arbres moteurs\footnote{Un arbre moteur est la partie centrale (liée au rotor) qui tourne lorsque le moteur est alimenté.}.
		\item Pourquoi Ité est toujours pardonné quelque soit ses bêtises~? Parce que \guillemotleft~Ce n’est pas grave, Ité\footnote{« Gravité ». Le niveau de cette blague est bien bas (à cause de la gravité~?)}~!~\guillemotright.
		\item Quel genre de musique le Fer, le Cuivre et le Plomb adorent-ils~? Le métal\footnote{Plutôt le \textit{Heavy Metal} pour le Plomb et l’Or.}.
		\item Quelle est la musique préférée des nombres relatifs~? Le Lac des signes\footnote{Minute culture~: c’est le Lac des Cygnes de Tchaïkovski. Et que personne ne me reproche de vous faire apprendre des choses fausses~!}.
		\item Un système asservi ne voyage pas~: il va voir du PI\footnote{Ou Proportionnel Intégrateur, un super correcteur pour ces systèmes}. 
		\item Deux électrons avec le même triplet de nombres quantiques mais ayant un spin opposé~:
		\begin{itemize}
			\item[-] Je veux changer de spin car tu es mon modèle~!
			\item[-] Ooooh\ldots Sois Pauli\footnote{Pauli, c’est le gars qui a dit que deux électrons ne pouvaient pas avoir le même quadruplet de nombres quantiques (donc le même spin). \\Ok, j’avoue, il est très difficile de voir un électron changer de spin mais qui a déjà vu deux électrons parler~?}~!
		\end{itemize}
		\item Pourquoi un mathématicien imprime-t-il ses photos à moitié~? Parce qu’il fait un développement limité.
		\item Quel est le comble pour un scientifique~? De mater Mathique et que cette dernière ne le calcule même pas.
		\item Un logicien a fait une chaise à bascules~: il l'a appelé JK\footnote{Les bascules JK sont utilisées pour réaliser de nombreux circuits logiques en élec~: compteurs, etc.}.
		\item Quelle est la liaison en étude de mécanismes qui n’est jamais en retard~? La liaison ponctuelle\footnote{Aussi connu sous le nom de « liaison sphère plan ».}.
		\item Qui est le plus impacté par l’interdiction de la bigamie~? L’oxygène~: il a deux liaisons covalentes.
		\item Si un torseur\footnote{Définition d'un torseur~: Un torseur ${\mathcal {T}}$ est un champ de vecteurs équiprojectif défini sur un espace affine euclidien $\mathcal{E}$ de dimension 3.} participait à la Nouvelle Star, que chanterait-il~? Un champ vectoriel.	
		\item Quel est l’organe humain le plus fort en maths~? Le rein\footnote{Ils font des calculs rénaux -- les maths, pour certains, c’est nocif~!}. \\\textit{\underline{Variante}~: comment soigner un calcul rénal~? Avec une calculette.}
		\item Pourquoi les solutions ioniques coûtent-elles une fortune~? Parce qu’elles ont beaucoup de charges~! \\ \textit{\underline{Variante}~: les particules ionisées et le plasma.}
		%\item Deux suites géométriques discutent~: 
		%\begin{itemize}
		%	\item[-] T’as un joli $q$, tu sais ?
		%	\item[-] Oooh, ne perds pas ta raison\ldots
		%\end{itemize}
		\item Que met une aile d’avion quand elle a froid~? Une polaire\footnote{Une polaire d’aile, c’est la courbe du coefficient de portance en fonction du coefficient de trainée.}.
		\item Qu’est-ce qu’un nuage complexe~? Un iCloud. \\\textit{\underline{Variante}~: Qu’est-ce qu’un nuage imaginaire pur~?}
		\item On parle de réduction de matrices\footnote{Le fait de diagonaliser ou de trigonaliser des matrices.}. Mais pourquoi on ne parle jamais d’oxydation\footnote{NE POSEZ PAS CETTE QUESTION À VOTRE PROF D’ALGÈBRE~! C’est un jeu de mot par rapport à la chimie et les réactions d’oxydoréduction.} de matrices~?
		\item Que fait Newton lorsqu’il va à The Voice~? Il fait un champ Newtonien.
		\item Un jour, j’ai voulu faire une blague bidon à une bouteille d’hélium. Elle n’a même pas réagi\footnote{Hélium~: gaz inerte ou gaz rare~: n’intervient dans aucune réaction chimique.}.
		\item Que fait un éléphant impuissant~? Il barrit sans trique\footnote{Barycentrique}.
		\item Quelle est la marque préférée de tablettes pour les fonctions mathématiques~? $\arccos(x)$\footnote{Archos est une marque de tablettes tactiles (et de baladeurs MP3) française. Cocoricoooo~!}.
		\item Un solénoïde ne respire pas~: il $N$ spire et il $x$ spire\footnote{Un solénoïde est un enroulement constitué de plusieurs tours de cuivre ($N$ et $x$~: nombre de spires).}. 
		\item Qu’est-ce qu’un Pi ($\pi$) insomniaque~? Un pissenlit\footnote{Pissenlit = Pi ($\pi$) sans lit.}.
		\item Qu’est-ce qu’un triangle aux normes~? Un triangle ISOcelle\footnote{Tout comme les normes ISO - ex~: la norme ISO9001.}~!
		\item Quelle est le système le plus chaud à résoudre~? Le système de Cramer\footnote{Un système qui n’admet qu’une seule solution car ce système a \textit{n équations indépendantes} avec \textit{n inconnues}.}.
		\item Deux dioptres jouent au poker~: \guillemotleft~ Alors t’as des cartes\footnote{Notre fameux ami René Descartes~! Si vous ne le connaissez pas, demandez aux prismes~: ils en connaissent un \textit{rayon}}~?~\guillemotright
		\item Un facteur de qualité\footnote{Le facteur de qualité Q d’un filtre du second ordre est inversement proportionnel à ce que laisse passer ce filtre. Q grand $\rightarrow$ filtre sélectif (et donc de qualité).} est un type qui te donne de bonnes lettres.
		\item Pourquoi la fonction $f(x) = x^2 + 1$ est toujours positive~? Parce qu’elle a un large sourire\footnote{Cette fonction est une parabole qui a un beau sourire.}.
		\item Une électrode, quand elle suit les dernières tendances vestimentaires, elle n’est pas à la mode mais à l’anode\footnote{Électrode qui est le siège de l’oxydation dans une pile.}.
		%\item Quelle est la chanson d’Indila qui convient le mieux aux moteurs~? \guillemotleft~ \href{https://www.youtube.com/watch?v=F3wpq-i150c}{Tourner dans le vide} ~\guillemotright.
		\item Pourquoi une solution ne vous calcule jamais quand vous lui parlez~? Parce qu'elle est trop concentrée.
		\item Quelle est la personne qui a le plus de ressources~? C’est Nernst~: parce qu’il a du potentiel\footnote{Le potentiel de Nernst~: l’incontournable de la chimie~!}.
		\item Un polymère\footnote{Ou une maman polie (polie mère). \\Plus sérieusement, c’est une molécule assez grosse (puisque composée de macromolécules).}, c‘est une maman qui a de bonnes manières.
		\item Quelle est la droite la plus triste de l’espace ${(O,x,y,z)}$~? La droite $(D')$\footnote{La droite déprime~! Hahahaha~! \tiny{(pardon, c'est pas drôle)}}. 
		\item Que mettent les coordonnées d’un point du repère quand ils se caillent trop (en hiver)~? Ils mettent une polaire\footnote{Une polaire est une veste qui tient chaud.}.
		\item Quel est le signal qui sait le mieux se garer~? Le signal en créneau.
		\item Pourquoi les nombres réels sont-ils tordus~? Parce qu’ils ne sont pas droits comme un $i$\footnote{Cette blague n'est pas si complexe~: $i$ est un nombre complexe.}. \\
		\textit{\underline{Variante}~: Une prise électrique rend droit comme un} I.\footnote{Le courant est souvent noté I (pour l'intensité) et se mesure en Ampères.} 
		\item M. et Mme FAIT ont un fils. Comment s’appelle-t-il~? Gaspard\footnote{Pour Gaz parfait~! Ils ont aussi un fils qui s’appelle Lation mais ils ne veulent pas en parler\ldots}.
		\item Quel est le plat favori des télescopes~? Les lentilles.
		%\item Si un torseur couple doit aller en cours, comment devra-t-il s’habiller~? Avec un uniforme\footnote{Les torseurs couple sont uniformes (le même en tout point).}.
		\item Si un physicien t’amène boire un verre, pourquoi faut-il s’attendre à ce qu’il y ait beaucoup de monde~? Parce qu’il t’amènera dans un bar\footnote{Pour un physicien, un bar = 100 000 Pascals (oh, salut Pascal~!)}.
		\item Comment fait-on pour trouver un schéma de logique combinatoire en NOR\footnote{Une logique combinatoire dans laquelle on ne peut qu’utiliser que les connecteurs \textbf{NON} ainsi que \textbf{OU}. Différent de la \textbf{NAND} (\textbf{NON} ainsi que \textbf{ET}).}~? Avec une boussole\footnote{Au moins, on ne perd pas le Nord~!}.
		\item Un arbre, quand il a froid, il ne met pas une écharpe mais une écharde.
		\item Un archer, lorsqu’il fuit, ne prend pas la tangente~: il prend l’arc tangente.
		\item Qu’est-ce qu’une salle de classe avec 29 PSI~? Une classe à 2 bars\footnote{14.5 psi (Pounds per Square Inch) équivaut à 1 bar (environ 1 kg/cm²). Normal que tu aies la pression en prépa~!}.
		%\item Si on veut faire un filtre et qu'on est trop agité (ou trop bruité), qui faut-il appeler? Kalman\footnote{Avec le filtre de Kalm-an ! Normalement, ça calme et ça filtre certaines perturbations.}.
		\item Quel est le quartier marseillais le plus redouté par les électroniciens ? Le cours Ju'\footnote{Le Cours Julien est un quartier bien connu à Marseille et ailleurs (ex : pour sa vie nocturne animée). Et un cours-jus (ou court-circuit) est généralement une erreur qu'on préfère éviter en électronique.}.
	\end{enumerate}
	
	\chapter{Jeux de mots}
	Pas de questions, que des réponses et des \textit{puns}. \\ \\
	Quelle est la différence avec les autres rubriques~? Nous trouverons ici de nombreuses réflexions philosophiques et des remarques qui feront avancer le monde, sans oublier les dialogues hilarants entre des personnages insolites. 
	\newpage
	\begin{enumerate}
		\item Des journaux sont en train de se faire imprimer mais un journal bloque la rotative\footnote{Le journal avait sûrement bu pour engendrer un bourrage papier}. Donc le journal d'à côté lui dit~: «~Allez, presse-toi~!~».
		\\ \textit{\underline{Variante}~: Ça marche aussi avec des citrons~!}
		\item Deux allumettes discutent dans une forêt~:
		\begin{itemize}
			\item[-] Je suis amoureux d’un arbre~: comment lui dire~?
			\item[-] Bah, déclare-lui ta flamme. 
		\end{itemize}
		\textit{\underline{Variante}~: Avec une allumette et un jerrican d’essence.}
		\item Ton escalope~: Jean la sale\footnote{Comme Jean Lasalle.} et Patrick la poivre\footnote{Comme Patrick Poivre d'Arvor (PPDA).}].
		\item Un téléphone ne meurt pas~: il déclare forfait\footnote{Il abandonne. Mais vous savez, c’est un téléphone donc il a forcément un forfait~! Hahahahaha~!!}.			
		\item Un chat, pendant un rendez-vous amoureux~:
		\begin{itemize}
			\item[-] Tu hantes mes nuits, tu hantes mes rêves… Chaque jour ton absence me fait souffrir. Mais cette souffrance laisse place à un grand apaisement quand je te vois\footnote{Je n'ai jamais dit que j'étais bon pour les déclarations (sauf pour les déclarations d'impôts).}. Je t’aime.
			\item[-] Tu sais, tu n’es pas mon genre mais nous pouvons rester amis. Désolé.
			\item[-] Mais pourtant on était félin\footnote{Ils étaient faits l'un pour l'autre... Tristesse~!} pour l’autre~!
		\end{itemize}			
		\item En fait, un smiley, il rit jaune~!
		\item Adriana Karembeu, quand elle veut bien s’habiller, elle ne se met pas sur son 31 mais sur son 42\footnote{Elle se met sur son Karembeu~!}.
		\item Deux chaussures discutent~:
		\begin{itemize}
			\item[-] Oh non~! J'ai marché sur une crotte\footnote{C'est pas très spontané mais il peut y avoir des enfants qui lisent~!}~! 
			\item[-] J'en n’ai rien à cirer. 
		\end{itemize}
		\item « Je ne suis pas devin, je suis quarante\footnote{$2\times20 = 40$.}!~»
		\item «~Didier des villes et Didier Deschamps~»\footnote{Une variante de la fable \textit{Le Rat des Villes et le Rat des Champs.}}.
		\item La mer~: le paradis des alcooliques car c’est le seul endroit où il y a un bar tous les dix mètres\footnote{En plongée, la pression augmente de 1 bar (ou 1000 hPa) tous les 10 mètres.}.
		\item Une vache ne colorie pas avec des crayons mais avec des trayons\footnote{Partie de la vache qui sert à la traite.}.
		%\item Une tomate, chez un docteur, ne fait pas un check-up mais un ketch-up.
		\item Plusieurs tasses à café sont sur le point de faire un 100 mètres. L’arbitre~: «~À vos marcs~!~»
		\item Deux pneus en vacances discutent~:
		\begin{itemize}
			\item[-] On va aller aux criques~! C'est trop bien, non~?
			\item[-] Bof, je vais encore me faire démonter\footnote{Ah, on dirait que ce pneu a mal vécu son démontage, qui a été fait avec un cric.}... »
		\end{itemize}			
		\item Einstein, quand il a faim, il ne mange pas à la cantine mais à la quantique\footnote{‘Fin, tout est relatif~!}.
		\item Un philosophe ne compte pas mais il Kant\footnote{Comme Emmanuel Kant, un grand philosophe.}.
		\item Que dit un \oe{}uf quand on le met dans le frigo~? « Je me caille ici~! ».
		\item Salamèche et sa mère sont dans la forêt. En le voyant jouer dans le parc, elle lui dit~: «~Arrête de jouer avec le feu~!~».			
		\item Un poète qui s'est pris un coup n'est pas sonné mais il est sonnet\footnote{Un sonnet est une forme de poésie avec deux quatrains et deux tercets.}.
		\item Deux boomerangs discutent. Lorsque l’un annonce le décès de son copain (à cause d’un incendie), l’autre répond~: «~J’en reviens pas~!~».
		\item La carpe a aussi son jour~: Carpe diem\footnote{Diem = le jour en latin. C'est une expression qui nous recommande de \guillemotleft cueillir le jour \guillemotright, soit de profiter du jour présent.}.
		\item Des glaçons pendant une photo de groupe. Le photographe~: « Dites \textit{Freeze}\footnote{Cette blague \textit{freeze} le ridicule, n'est-ce pas~?}~! ».
		\item Il existe deux types de champs~: les bas-champs et les Auchan.
		\item En fait, les hooligans français se tapent dessus pour soutenir les Bleus\footnote{Soutenir les Bleus en se faisant des bleus. Le Grand Schtroumpf aime ça.} pour l’Euro~!
		\item J’ai appelé mon sèche-cheveux Vection. Comme ça, quand il me brûle, je peux lui dire « T’es con, Vection\footnote{Parce que la chaleur est transmise par convection avec un sèche cheveux.}~! »
		\item Deux ampoules sont sur un luminaire et discutent à propos de leur nouvelle voisine~:
		\begin{itemize}
			\item[-] Alors, comment tu la trouves~?
			\item[-] Oh mon Dieu, qu’est-ce qu’elle est LED\footnote{...en plus de ne pas être une lumière.}~!
		\end{itemize}
		\item Que dit une chaussure à ses acolytes quand elle part pour le week-end ? «~À la semelle prochaine~!~».		
		\item Deux fœtus dans un ventre~: « Tu m’as encore donné un coup de pied~! C’est placenta\footnote{Placenta = pas sympa.}~!~».
		%\item Si les bonbons n'étaient pas bons, ça s'appellerait des mauvaismauvais~! 
		\item Dans un sac à main, c'est un stylo qui crie sur un agenda qui est à côté\footnote{Il y a de l'ambiance dans ces petits sacs à main~!}~: \guillemotleft~Tu m’as encore volé de l’encre~! Je t’ai pris la main dans le sac~!~\guillemotright
		\item C’est l’Océan Atlantique qui dit au Gulf Stream~:
		\begin{itemize}	
			\item[-] Tu le savais que le réchauffement climatique, ce n’était pas bon pour moi~?
			\item[-] Ah bon~? Je n’étais pas au courant\footnote{Le Gulf Stream est un courant marin chaud qui traverse l’Océan Atlantique.}~!
		\end{itemize}
		\item Deux enfants, à la sortie des cours. 
		\begin{itemize}
			\item[-] À propos de ton heure de retenue hier~: ta grand-mère n’a pas eu une dent contre toi~?
			\item[-] Bah non~: elle porte un dentier\footnote{Donc non, elle n'a pas de dents contre lui~! Ouf~!}~!  
		\end{itemize}
		\item Deux citrons discutent au rayon fruits de Carrefour. L’un d’eux raconte une blague assez marrante (lisez ce recueil, ça ne manque pas~!). En voyant la réaction de son copain, le premier citron lui demande~: «~Mais pourquoi tu ris jaune~?~».
		\item C’est une photo qui va chez le coiffeur. Elle demande~: «~Bonjour, je voudrais me couper les TIFF\footnote{TIFF est un format d’image, comme le JPEG mais avec quelques différences (ce dernier est plus compressé).}~».
		\item Deux fils électriques après leur heure de conduite~: 
		\begin{itemize}
			\item[-] Alors, t’as bien conduit~?
			\item[-] Ouais, ça va~: on m’a juste balancé du jus et je n’ai pas trop résisté. Et toi~?
			\item[-] Laisse tomber, j’étais à la masse… 
		\end{itemize}
		\item Spiderman savait utiliser la Toile\footnote{La Toile d’araignée~!} avant que ça devienne à la mode\footnote{Et il avait un super réseau (\textit{\textbf{net}work})~!}~!
		\item \guillemotleft~Arrête avec tes Sean Connery\footnote{Une manière \guillemotleft~polie~\guillemotright de demander d'arrêter d'entendre des inepties (ou des conneries).}~\guillemotright~!
		\item Tu connais le gars qui avait dix bras ? Pour lui, c'est dimanche\footnote{Dix manches.} tous les jours !
	\end{enumerate}
	
	\chapter{Notre ami Toto}
	Toto est ce personnage qui arrive à penser comme aucun autre. Ses réactions et ses réponses sont tellement surprenantes que vous vous plierez de rire (sauf si vous êtes un professeur et que vous l'avez en tant qu'élève). \\ \\
	Voici son histoire. \textit{DUM DUM}. 
	\newpage
	\begin{enumerate}
		\item À l’école, la classe de Toto fait de la géométrie. Tout le monde a son compas sauf Toto. Donc la prof le questionne~:
		\begin{itemize}
			\item[-] Pourquoi tu n’as pas apporté ton compas~?
			\item[-] Parce qu’on m’a dit que j’avais le compas dans l’œil~! 
		\end{itemize}
		
		\item En maths, la prof interroge Toto, assez distrait~: 
		\begin{itemize}
			\item[-] Toto, comment note-t-on la fonction exponentielle~?
			\item[-] Euhhhhh...
			\item[-] Oui, bien~! Bonne réponse~!
		\end{itemize}
		\textit{\underline{Variante} (plus facile à comprendre)~: Quelle est la 5\textsuperscript{ème} lettre de l’alphabet~?}
		
		\item En maths, la prof interroge Toto, toujours aussi distrait~:
		\begin{itemize}
			\item[-] Toi à moitié endormi~: combien vaut $ln(e)$~?
			\item[-] Heiiin~?
			\item[-] Oui, c’est très bien\footnote{$ln(e) = 1$.}~!
		\end{itemize}
		
		\item C’est la mère de Toto qui surprend son fils en train de mettre une couverture sur son livret A~:
		\begin{itemize}
			\item[-] Pourquoi tu mets une couverture sur ton livret\footnote{À une époque, les livrets d'épargne ressemblaient vraiment à un livret papier.}~? Tu as peur qu'il attrape froid~?
			\item[-] C’est pour que mon compte ne soit pas à découvert~!
		\end{itemize}
		
		\item C’est Toto qui se fait arrêter par un policier~:
		\begin{itemize}
			\item[-] Monsieur, papiers du véhicule s'il vous plaît~!
			\item[-] Attendez, je vous les passe mais vous pouvez me tenir ma bière~?
		\end{itemize}
		
		\item C’est le père de Toto qui surprend son fils en train de mettre ses cours de maths et de physique dans de l’eau bouillante~:
		\begin{itemize}
			\item[-] Mais qu’est-ce que tu fais~? Tu es censé mettre les pâtes dedans, pas tes cours~!
			\item[-] Je sais, j’attends que la science infuse~!
		\end{itemize}
		
		\item Aujourd’hui, c’est repas de famille~! Après que les convives ont demandé de l’eau, Toto va en chercher. Les minutes passent, toujours rien. Donc sa mère vient le voir dans la cuisine et le surprend avec une bouteille d’eau et un rouleau de pâtisserie. Sa mère~:  
		\begin{itemize}
			\item[-] À quoi tu joues~? On t'a demandé de l'eau, pas un rouleau~!
			\item[-] Bah on m’a demandé de l’eau plate~!
		\end{itemize}
		
		\item C’est Toto qui ne travaille pas depuis voilà deux semaines. Les notes sont en chute libre\footnote{Déjà qu'elles ne furent pas très élevées...}. Son père lui passe alors un savon~:
		\begin{itemize}
			\item[-] Pourquoi tu ne bosses pas~? Les concours sont dans deux semaines et tu vas te vautrer si tu continues~!
			\item[-] Bah, mon crayon ne fait rien et pourtant il a eu les Mines\footnote{Un étudiant en classes préparatoires peut être accepté à l'École des Mines grâce au concours des Mines (les épreuves sont réputées être difficiles). Pour y arriver, il faut avoir un bon classement à l'issue des épreuves écrites ET orales. \\Le crayon n'a pas suivi cette voie vu qu'il a déjà une mine (mine de rien).}~!
		\end{itemize}
		
		\item 18 juin, cinq heures du matin. Toto se réveille en sursaut et demande à sa mère~:
		\begin{itemize}
			\item[ ] \guillemotleft~Maman, maman~! Faut absolument que j’aille à Castorama pour aller acheter une pelle~!
			\item[-] Ça ne peut pas attendre demain~? », dit sa mère, encore endormie.
			\item[-] « Bah non, parce que sinon je raterai la pelle du 18 juin~! » 
		\end{itemize}
		
		\item C’est Toto qui va dans un cours de musique avec un bloc de glace qui a la forme d’un piano. Le prof de musique s’approche de lui et demande~:
		\begin{itemize}
			\item[-] Monsieur, pourquoi vous vous pointez avec un bloc de glace dans mon cours de musique~? Vous n’êtes pas à un concours de sculpture.
			\item[-] Ah, parce que je voulais participer à votre cours de piano aqueux\footnote{Toto a raison~: le piano aqueux coûte bien moins cher qu’un piano à queue~! En plus, il aura moins de mal à briser la glace avec ses camarades~!}~!
		\end{itemize}
		
		\item Toto, ingénieur en école, doit concevoir un frein. Il connecte alors les disques à des batteries. Son prof encadrant, surpris, lui pose des questions~:
		\begin{itemize}
			\item[-] Pourquoi branchez-vous des batteries à ces disques~? Cela ne sert à rien sans l'électronique derrière~!
			\item[-] C'est simple~: c'est pour que le véhicule pile\footnote{C'est bien pour un freinage d'urgence. Quoique, je ne ferai pas confiance aux produits conçus par Toto.}.
		\end{itemize}
		
		\item C’est Toto qui veut se faire un thé. Pour cela, il remplit sa tasse d’eau (logique), met son sachet de thé et... met le tout dans le coin de la pièce. Son colloc, étonné, lui dit~:
		\begin{itemize}
			\item[-] Tu sais, ça ne sert à rien de mettre ta tasse dans un coin. Il faut chauffer l’eau au préalable~! »
			\item[-] Je sais. Il faut que le thé soit à 90°. Et ce coin fait 90°.  
		\end{itemize}
		
		\item C’est Toto qui va à la banque. Il se dirige vers l’accueil~:
		\begin{itemize}
			\item[-] Bonjour, avez-vous des pansements~?
			\item[-] Ici, c’est une banque~! La pharmacie se trouve juste en face~!
			\item[-] Je sais mais en allant au distributeur, j'ai eu plein de petites coupures\footnote{Des billets de banque.} à cause de vous~!
		\end{itemize}
		
		\item Toto est invité à un gala. Pour l'occasion, il se met sur son 31 et... il amène du fromage (!). Le staff lui demande\footnote{C'est fou comme ils en font tout un fromage~!}~:
		\begin{itemize}
			\item[-] Toto, à quoi tu joues~? C'est une soirée élégante, pas une soirée raclette~!
			\item[-] C'est pour la danse~: ce fromage roquefort\footnote{Il rock fort~!}.
		\end{itemize}
		
		\item Toto va à la boucherie. Lorsqu'il a acheté sa viande, il ouvre son sac et vide une boîte de thé sur ses steaks. Le boucher, assistant à cette scène loufoque, l'interroge sur ses motivations~:
		\begin{itemize}
			\item[-] Mais pourquoi~? Vous avez perdu la tête~? La viande était déjà parfu...
			\item[-] Tout simplement parce qu'on m'a dit que la viande est une pro-théine\footnote{Et si vous vous demandez~: non, une protéine n'est pas forcément pro ou anti théine. Juste évitez de boire du thé après avoir mangé de la viande rouge car cela empêche l'absorption de fer. }. 
		\end{itemize}
		
		\item Pourquoi l’imprimante de Toto ne marche jamais et sent la pâtisserie~? Parce qu’il a mis mille-feuille\footnote{En plus, ça finira sûrement avec un bourrage~: j'en suis déjà à mon cinquième mille-feuille en quelques jours~!}.
	\end{enumerate}
	
	\chapter{Voyages}
    Envie de voyager~? Alors prenez votre passeport\footnote{Veuillez vérifier la date de validité de votre passeport}, vos valises\footnote{Maximum : 23 kg et d'un volume inférieur à XXX L et ne doit pas contenir de produits dangereux} et en route~!
    \\Nul besoin d'agence de voyage car nous avons toutes les cartes \sout{du monde} en main ! 
    
	\newpage
    \begin{enumerate}
        \item Pourquoi sommes-nous à l'étroit si nous faisons Paris-Strasbourg en voiture\footnote{L’autoroute A4 relie Paris et Strasbourg (et plein d’autres villes aussi).}~? Parce que l'A4 ne fait que 21 x 29.7 cm~!
		\item Pourquoi l’Espagne n’a-t-elle pas de poils~? Car il a un Cordoue.
		\item Pourquoi il n'est pas bon d'être vendeur en Australie~? Parce que là-bas, ils vendent à Perth. 
		%\item Comment payait-on par chèques avant 1993~? On payait par chècoslaves\footnote{Référence à la Tchécoslovaquie qui disparut en 1993 et qui est devenue la Rép. Tchèque (et la Slovaquie).}.
		%\item Pourquoi les habitantes de Six-Fours ont-elles toujours chaud~? Parce que ce sont des Sixfournaises\footnote{Une magnifique ville non loin de Toulon.}. %
		\item Qu’est-ce qu’une personne de petite taille qui a une peau soyeuse~? Un naindoux\footnote{Hindou.}.
		\item Que met le Soleil lorsque son pantalon est trop large~? Une \textit{Sun Belt}\footnote{C'est une zone géographique au sud des États-Unis (sous le 36\textsuperscript{e} parallèle), caractérisée par un climat doux et ensoleillé.}.
		\item Pourquoi les agents du renseignement israéliens sont-ils de mauvaise humeur\footnote{T'as déjà vu ces bonhommes sourire~? Voilà~!}~? Parce qu’ils sont d’humeur Mossad.
		%\item Pourquoi certains Indiens ont-ils un peu de ventre~? Parce qu’ils ont le ventre Bombay\footnote{C’est une grande ville en Inde (officiellement connu sous le nom de Mumbai).}. 
		%\item Pourquoi la Slovaquie est-elle plus riche que la Rép. Tchèque~? Parce qu’ils faisaient souvent des Tchécoslovaques\footnote{Des chèques aux Slovaques (et pour les infos sur la richesse de ces deux pays, ne prenez pas le contenu d'un recueil de blagues pour \textbf{argent} comptant~!)}.
		\item Pourquoi les étudiants des Mines d’Alès ne sont-ils pas au top de leur forme~? Parce qu’ils ont une petite Mine\footnote{Pour intégrer les Mines d’Alès, il faut réussir le concours des « petites Mines » (les Grandes Mines correspondent aux Mines de Paris, l’École des Ponts, etc).}~!
		\item Quel est le pays qui a le meilleur pouvoir d’achat~? \\L’Azerbaïdjan~: ils achètent tout à Bakou\footnote{Bakou est la capitale de ce pays. Et concernant le pouvoir d'achat dans ce pays, ce n'est pas réellement prouvé.}.
		\item Quel serait le comble de la ville de Berlin~? D'être sur Facebook\footnote{ Et donc un mur -- Ce calembour a mal vieilli.}. 
		\item Pourquoi les grands-pères en Russie sont-ils aussi vieux~? Car ce sont des papis russes\footnote{Des papyrus. Hahahaha~!!!!}.
		\item Pourquoi le Sri Lanka se développe bien plus rapidement qu'avant~? Parce qu'avant, Ceylon\footnote{\guillemotleft C'est long \guillemotright. Il n'y a pas de sources géographiques derrière cette constatation}. 
		\item Quelle est la ville dans laquelle garder la ligne est le plus aisé ? Agen\footnote{Pour pouvoir être à jeun !}.
		\item Pourquoi peut-on parfois observer une piscine au bureau du Premier ministre canadien ? Parce qu'il y a \guillemotleft~juste un trou d'eau \guillemotright \footnote{Justin Trudeau} !
        \item La Syrie est un pays bizarre~: ses citoyens ne scient rien alors qu'ils habitent dans une scierie. 
        \item $\zeta\zeta\zeta\zeta\zeta$ $\zeta\zeta\zeta\zeta\zeta$ $\zeta\zeta\zeta\zeta\zeta$ $\zeta\zeta\zeta\zeta\zeta$ $\zeta\zeta\zeta\zeta\zeta$ \\ 
        $\zeta\zeta\zeta\zeta\zeta$ $\zeta\zeta\zeta\zeta\zeta$ $\zeta\zeta\zeta\zeta\zeta$ $\zeta\zeta\zeta\zeta\zeta$  $\zeta\zeta\zeta\zeta\zeta$ \\
        ↑ Les Zêta-Unis\footnote{Et en plus, il y a 50 zêtas~! Quelle coïncidence~! (référence aux USA et à ses 50 États).}.			
        \item Toto et sa grande sœur sont en voyage à Londres. Lorsque Toto rentre dans sa chambre tout ensanglanté, sa sœur, surprise, lui demande~:
        \begin{itemize}
            \item[-] Oh mon Dieu, mais qu'as-tu encore fait~?
            \item[-] J’étais à la boulangerie et j’avais demandé « Je \textit{want some} pain, \textit{please}. »\footnote{\textit{Pain} en anglais = douleur. Aïe~! D'où l'importance des cours d'anglais.}.
        \end{itemize}
    \end{enumerate}

	
	\chapter{Humour noir}
	Soyons d'accord~: je ne pense absolument pas aux propos tenus dans ces pages suivantes. Certaines blagues peuvent être de mauvais goût et je m'en excuse si c'est le cas. Je respecte les personnalités/nations citées, ainsi que les accomplissements réalisés par ces dernières. \\ 
	Si vous racontez ces blagues à d'autres personnes, elles ne feront pas rire tout le monde, donc faites preuve de discernement avant. 
	\newpage
	\begin{enumerate}
		\item Pendant les attentats du 11 Septembre, pourquoi il aurait fallu filmer d’en bas~? Pour entendre les gens chanter «~\textit{It’s raining men}~».
		\item Comment un vampire fait-il pour se prendre une cuite~? Il pompe le sang d’un ivrogne. 
		\\\textit{\underline{Variante}~: Remplacer «~vampire~» par Edward (de \underline{Twilight}).}
		\item Quel est le point commun entre Lady Diana et un téléphone~? Les deux ne passent pas sous un tunnel\footnote{Lady Diana a perdu la vie dans un accident au tunnel du pont de l'Alma (à Paris). Et un téléphone n'a pas de réseau sous un tunnel.}.
		\item Dans Titanic, c’est Rose qui aurait dû couler et non Jack~: au moins, ça aurait été un film à l’eau de Rose.
		\item Quel est le jeu préféré des croque-morts~? Le cadavre exquis.
		\\\textit{\underline{Variante}~: Marche aussi avec les charognes.}
		\item Si un gars s’immole par le feu, est-ce un one man chaud~?
		%\item Ariel Sharon n’était pas dans la matrice mais dans la comatrice\footnote{La comatrice de A est une matrice constitué des cofacteurs de A (notion d'algèbre linéaire).}\textsuperscript{,}\footnote{A. Sharon a fait un grave AVC en 2006. Pendant huit ans, il est resté dans un coma dont il ne s'est pas réveillé.}. 
		\item Pourquoi David Bowie, Prince et Frank Sinatra sont-ils morts en l’espace de quelques mois~? Parce que Dieu a découvert la musique dans le Cloud\footnote{Si les artistes sont au paradis, ils sont au ciel (et dont dans le cloud). Paix à leur âme.}.
		\item Pourquoi les Japonais peuvent super bien capter la radio autour de Fukushima~? Parce que c'est une zone radio-active\footnote{Surtout après l'incident de la centrale de Fukushima en 2011.}. 
		\item Quel est le point commun entre un livreur de pizza et une ambulance~? S’ils arrivent en retard, le produit sera froid\footnote{Si la pizza est froide, ce n'est pas trop grave. Mais si un patient l'est, c'est un peu plus grave.}. 
	\end{enumerate}
	
	\chapter{Autres blagues}
	Attention, NSFW\footnote{Not Safe For Work}~! \\ \\
	Pensez à planquer ces quelques pages lors d'un repas avec la belle famille ou si vous voulez faire rire des enfants\footnote{Astuce~: ce chapitre et le suivant ont été stratégiquement placés à la fin du recueil. Ainsi, en milieu sensible, il vous suffira simplement d'arracher \sout{(et de détruire)} ces dernières pages.}. \\Et faites attention au public ciblé~! 
	\\ \\L'auteur n'est pas responsable de ces pages~: elles ont été rédigées toutes seules, comme par magie~!
	\newpage
	\begin{enumerate}
		\item \texttt{Pourquoi l’informatique a contribué au porno~? \\Car on compte en bits.}
		\item Vous savez, le gars qui avait trois sexes~: il a réussi à trouver la 3G\footnote{Lors de la rédaction de ce trait d'esprit, la 4G n'existait pas (et la 5G n'était même pas en projet)~!} avec sa femme\footnote{Parce que notre homme a pu trouver le point G trois fois... Donc 3G~! Nous ne ferons pas de blagues sur le haut débit cependant}~!
		\item Fait divers~: une main droite porte plainte pour viol répété depuis l’âge de 14 ans\footnote{Quand une personne se consacre à son plaisir solitaire, a-t-elle demandé à sa main avant~?}.
		\item «~J’ai jamais compris pourquoi on me file toujours un mouchoir alors que le film n’est absolument pas triste~!~» -- Un pénis
		\item What’s the worst for a periodic table? To have its period\footnote{Hopefully, that happens only once a month.}.
		\item \texttt{Pourquoi un PC est le roi des plan à plusieurs~? Avec ses multiples ports USB, on peut rentrer plein des clefs\footnote{Et en plus, les débits se comptent en mégabits par seconde... Wow~!}~!~}
		\item Pourquoi un proxénète est-il un expert pour Excel~? Parce qu’il connait bien les macros\footnote{Aaaaah mince, mon correcteur a corrigé maquereaux par macros~!}.
		\\ \textit{\underline{Variante}~: Marche aussi avec la photographie macro.}			
		\item Qui sait le mieux lire sur les lèvres des gens~? Un gynécologue.
		\item Est-ce qu'une sodomie est un type de connexion par fibre\footnote{Quand des fibres (alimentaires) se font digérer, elles finissent dans les intestins puis dans le colon. Oui, cette blague est profonde~!}~?
		\item Deux tablettes de chocolat font l’amour. Qui est le mâle~? C’est celui qui a les noisettes. 
		\item Un homme impuissant n’a ni acide nitrique\footnote{Ni acide, ni trique. C'st la débandade~!}.
		\item If you strike something with your dick, you will have an hard-on collider\footnote{You have an hard-on when you see something... well... exciting (as a man). Reference to the LHC (Large Hadron Collider).}. 
		\item Deux pénis dans une plage de nudistes~: 
		\begin{itemize}
			\item[-] Tiens, elle est belle~: comment tu la trouves~?
			\item[-] Elle me gonfle~!~
		\end{itemize}
		\item Deux gars dans un bar~:
		\begin{itemize}
			\item[-] Hey, t’as vu le film samedi soir sur Canal+\footnote{C'était à l'époque où Canal+ diffusait un film X le samedi soir.}~?
			\item[-] Non, je ne voulais pas~: ça risquait de partir en couilles. 
		\end{itemize}
		\textit{\underline{Variante}~: «~Est-ce que t’as regardé La Cambrioleuse hier soir~?~»}
		\item Quel est le groupe de musique qui possède la plus imposante poitrine~? Boney M\footnote{Parce qu'ils mettent des soutifs avec un bonnet M~! \textsuperscript{[ref. needed]} }.
		\item Que fait une porteuse qui est excitée~? Elle bande\footnote{Pour bande passante de la porteuse, bien évidemment~!}.
		\item Que dit une carotte qui « rend visite » à une orange~? « Que tu as de belles lèvres pulpeuses\footnote{En même temps, une orange a pas mal de pulpes.}~! »
		\item Une personne castrée ne connaît plus les opérateurs booléens\footnote{Vu que ce sont les opérateurs de Boole.}.
		\item Quelle est la personne la plus frustrée~? Geisha parce qu’elle a les boules\footnote{Les boules de Geisha sont utilisées pour certains actes.}.
		\item Quel est le système de mesures qui a le plus d’érections~? Le système métrique\footnote{Le système \guillemotleft mes triqes \guillemotright. Même si le système impérial n'est pas en reste avec ses verges de 91.44~cm~!}.
		\item Pourquoi le cours du CAC 40 augmente en hiver~? Parce que les bourses montent\footnote{Les hommes connaissent bien ces fluctuations~!}. 
		\item \texttt{Que font des bases de données pour générer encore plus de données~? Une data-baise\footnote{\textit{Database} est un mot anglais signifiant \guillemotleft~Base de données~\guillemotright. Et baiser permet de se reproduire. }.}
		\item Quelle est la bourse la plus perverse~? Le Nikkei\footnote{Avouns-le, ce marché \textbf{nique} le game~!}.
		\item Est-ce que se masturber dans un livre du code civil est considéré comme violer la loi~?
		\item Qu’est-ce qu’un godemichet~? Un sex-appeal\footnote{Un sexe à piles (pour ceux qui ont de bonnes \textit{features}).}. 
		\item If an uncircumcised\footnote{Someone with a foreskin.} guy goes in a brothel, he will give automatically a tip\footnote{One of the numerous words to describe a foreskin... or money given as a thank you after a service.}. 
		\item En fait, la logique combinatoire, c’est une histoire de Boole et de bits.
	\end{enumerate}
	
	\chapter{Le petit mot de la fin}
	Quelques petits extras et des informations complémentaires. 
	\section{Infos légales}
	\textit{Ce livret est sous licence Creative Commons CC BY-NC 4.0. En gros, faites-en ce que vous voulez (partager, rigoler, annoter, imprimer, copier, manger, brûler, faire des avions en papier, etc) tant que vous n’en faites pas un usage commercial sans mon autorisation (et si vous le partagez en ligne, précisez l'auteur original~: \underline{Wissam S.} ou \underline{Bob Sleigh} (C'est mon surnom)). Merci.}  
	\newpage
	\section*{Résumé}
	Hihihihahahohoho~! Hihihihahahohoho~! Hahahahahaha~! \\
	Hihihihahahohoho~! Hihihihahahohoho~! Hahahahahaha~! \\
	Hihihihahahohoho~! Hihihihahahohoho~! Hahahahahaha~! \\
	Hihihihahahohoho~! Hihihihahahohoho~! Hahahahahaha~! \\
	Hihihihahahohoho~! Hihihihahahohoho~! Hahahahahaha~! \\
	Hihihihahahohoho~! Hihihihahahohoho~! Hahahahahaha~! \\
	Hihihihahahohoho~! Hihihihahahohoho~! Hahahahahaha~! \\
	Hihihihahahohoho~! Hihihihahahohoho~! Hahahahahaha~! \\
	Hihihihahahohoho~! Hihihihahahohoho~! Hahahahahaha~! \\
	Hihihihahahohoho~! Hihihihahahohoho~! Hahahahahaha~! \\
	Hihihihahahohoho~! Hihihihahahohoho~! Hahahahahaha~! \\
	Hihihihahahohoho~! Hihihihahahohoho~! Hahahahahaha~! \\
	Hihihihahahohoho~! Hihihihahahohoho~! Hahahahahaha~! \\
	Hihihihahahohoho~! Hihihihahahohoho~! Hahahahahaha~! \\
	Hihihihahahohoho~! Hihihihahahohoho~! Hahahahahaha~! \\
	Hihihihahahohoho~! Hihihihahahohoho~! Hahahahahaha~! \\
	Hihihihahahohoho~! Hihihihahahohoho~! Hahahahahaha~! \\
	Hihihihahahohoho~! Hihihihahahohoho~! Hahahahahaha~! \\
	Hihihihahahohoho~! Hihihihahahohoho~! Hahahahahaha~! \\
	Hihihihahahohoho~! Hihihihahahohoho~! Hahahahahaha~! \\
	Hihihihahahohoho~! Hihihihahahohoho~! Hahahahahaha~! \\
	Hihihihahahohoho~! Hihihihahahohoho~! Hahahahahaha~! \\
	Hihihihahahohoho~! Hihihihahahohoho~! Hahahahahaha~! \\
	Hihihihahahohoho~! Hihihihahahohoho~! Hahahahahaha~! \\
	Hihihihahahohoho~! Hihihihahahohoho~! Hahahahahaha~! \\
	Hihihihahahohoho~! Hihihihahahohoho~! Hahahahahaha~! \\
	Hihihihahahohoho~! Hihihihahahohoho~! Hahahahahaha~! \\
	Hihihihahahohoho~! Hihihihahahohoho~! Hahahahahaha~! \\ \\
	Haha\footnote{MDR}~?
	
\end{document}